\documentclass[a4paper,11pt]{article}
\usepackage{tabularx}
\usepackage{geometry}
\usepackage{xcolor}
\usepackage{physics}
\usepackage[most]{tcolorbox}
\usepackage{amsmath}
\usepackage{amssymb}
\usepackage{multicol}
\usepackage{float}
\usepackage{graphicx}
\usepackage{fancybox}
\usepackage{caption}
\usepackage{amsthm}
\usepackage{url}
\tcbuselibrary{breakable}
\geometry{left=1.4cm, top=0.8cm, right=1.2cm, bottom=1cm}

% Entorno para ejercicios con recuadro azul
\newtcolorbox{ejercicio}[1]{
    colback=white!8, colframe=blue!5!black,
    boxrule=0.75pt, arc=5pt, boxsep=10pt,
    title={\textbf{Ejercicio #1}}, fonttitle=\bfseries,
    breakable,
    before upper={\parindent15pt} % Añade sangría al texto
}

% Entorno para demostraciones con recuadro verde
\newtcolorbox{demostracion}[1]{
    colback=blue!5, colframe=gray!50!black,
    boxrule=0.75pt, arc=5pt, boxsep=10pt,
    title={\textbf{#1}}, fonttitle=\bfseries,
    breakable
}

\providecommand{\abs}[1]{\ensuremath{\left|#1\right|}}
\providecommand{\paren}[1]{\ensuremath{\left(#1\right)}}
\providecommand{\corche}[1]{\ensuremath{\left[#1\right]}}
\newcommand{\E}[1]{\mathbb{E}\left[#1\right]} % Esperanza
\newcommand{\LG}[1]{\mathcal{L}\left[#1\right]} % Transformada de Laplace
\newcommand{\LGV}[1]{\mathbb{L}\left[#1\right]} %
\newcommand{\V}[1]{\mathbb{V}\left[#1\right]} % Varianza
\newcommand{\COV}[2]{\mathbb{COV}\left[#1, #2\right]} % Covarianza
\NewDocumentCommand{\VA}{m}{\mathbf{#1}}

%----------HEADING-----------------
\begin{document}

\begin{tcolorbox}[colback=gray!10, colframe=black, boxrule=0.5pt, arc=5pt, boxsep=5pt]
\begin{tabularx}{\linewidth}{X r}
  \begin{tabular}[t]{@{}l@{}}
    \textbf{Introducción a la Ciencia de Datos} \\
    \textbf{Tarea I}
  \end{tabular}
  &\textbf{Antonio Barragán}\\
  &\textbf{Hazel Sánchez}\\
   & \textbf{Omar García Ramos} \\
\end{tabularx}
\end{tcolorbox}



\section*{Descripción de la base de datos}

El conjunto de datos está organizado en un \texttt{DataFrame} con \textbf{415 observaciones} (filas) y \textbf{26 variables} (columnas).
Contiene información sobre sitios de muestreo de árboles en Europa y sus registros isotópicos ($\delta^{13}C$) a lo largo de varios años.
Los valores faltantes se indican como \texttt{NA}.

\subsection*{Estructura de las variables}

\begin{table}[H]
    \centering
    \caption{Estructura del DataFrame (\texttt{df.info()})}
    \label{tab:info}
    \begin{tabular}{|r|l|r|l|}
        \hline
        \# & Variable & Valores no nulos & Tipo de dato \\
        \hline
        0   & Site Code & 415 & object \\
        1   & BRO       & 111 & object \\
        2   & CAV       & 374 & object \\
        3   & CAZ       & 412 & object \\
        4   & COL       & 289 & object \\
        5   & DRA       & 235 & object \\
        6   & FON       & 292 & object \\
        7   & GUT       & 412 & object \\
        8   & ILO       & 412 & object \\
        9   & INA       & 412 & object \\
        10  & AHI       & 293 & object \\
        11  & LAI       & 201 & object \\
        12  & LIL       & 408 & object \\
        13  & LOC       & 264 & object \\
        14  & NIE1      & 413 & object \\
        15  & NIE2      & 413 & object \\
        16  & PAN       & 412 & object \\
        17  & PED       & 410 & object \\
        18  & POE       & 412 & object \\
        19  & REN       & 376 & object \\
        20  & SER       & 409 & object \\
        21  & SUW       & 414 & object \\
        22  & VIG       & 337 & object \\
        23  & VIN       & 159 & object \\
        24  & WIN       & 244 & object \\
        25  & WOB       & 408 & object \\
        \hline
    \end{tabular}
\end{table}


\subsection*{Clasificación de las variables según su escala de medición}

A continuación se presenta la clasificación de cada variable de la base de datos según su escala de medición. Se incluye una breve explicación del porqué de cada clasificación.

\begin{itemize}
    \item \textbf{Site Code}: Nominal.
    Se trata de un identificador de los sitios de muestreo, por lo que no posee un orden ni valor numérico significativo.

    \item \textbf{BRO}: Razón ($\delta$13C).
    Representa valores isotópicos medidos, donde el cero tiene un significado absoluto y las diferencias entre valores son significativas.

    \item Continuara...

\end{itemize}


\end{document}
